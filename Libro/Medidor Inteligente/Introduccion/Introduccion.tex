%Intro
\chapter{Introducción}\label{sec:capitulo1}
\thispagestyle{empty}

\begingroup
\rightskip0.5cm
\small

\endgroup

\section{Antecedentes}
En la actualidad la evolución de las ciudades inteligentes se ha vuelto un punto
de interés en la mayoría de los países. La integración de las nuevas tecnologías
que permiten medir variables que antes eran inimaginables han dado paso a la
creación de productos que son capaces de controlar sistemas que mejoran la
condición de vida de los ciudadanos y el medio ambiente. En vías de cuidar el
medio ambiente y una mejor calidad de vida se han comenzado a desarrollar
las Smart Grids, redes de generación distribuidas inteligentes que poseen la
capacidad de medir el flujo de potencia de manera bidireccional, característica
que antes no existía, y al mismo tiempo permiten controlar de una forma más
eficiente y limpia la distribución de energía. Para este fin, es necesario dotar
a las redes de distribución de inteligencia con medidores inteligentes, capaces
de dotar con precisión los datos del flujo de potencia en ambos sentidos de la
red en tiempo real.\\

\par Un claro ejemplo de esta cualidad, es la invención del automóvil, el cual
 llegó como solución al problema del transporte por tierra, de una forma más
 cómoda y práctica. Dicho problema no se resolvió hasta que el ingeniero alemán
 Karl Friedrich Benz creó el primer automóvil en 1885 \cite{cernuschi2005cuatro},
 abriendo de esta forma, las puertas a un nuevo mundo para la investigación.\\


\section{Justificación y Planteamiento del Problema}

En la actualidad, más de 1,25 millones de presonas mueren cada año como
consecuencia de accidentes de tránsito, y aproximadamente 50 millones
 sufren traumatismos no mortales, los cuales pueden llegar a producir
 alguna discapacidad \footnote{http://www.who.int/mediacentre/factsheets/fs358/es/}.
  Dichos siniestros son causados, en su mayoría, por la imprudencia del ser
   humano. Si no se toman medidas correctivas se espera que estas cifras tan
   alarmantes aumenten para el año 2030, de tal forma que se conviertan la
   séptima causa de muerte en el mundo. Es por eso que la organización de las
   naciones unidas, ONU (del inglés, \textit{Organization of United Nation})
   adoptó \textit{La agenda 2030} para el desarrollo sostenible, donde se
   espera que para el 2020 se disminuyan estos números a la mitad, a través de
   dsitintos planes.\\

\par Para apoyar estas soluciones que se pretenden poner en práctica, el
 desarrollo e implementación de los ITS juegan un papel muy importante,
 ya que los mismos buscan solventar los fallos del ser humano, ya sea mediante
  acciones pasivas, como lo puede ser una simple notificación al conductor de
  alguna falla o infracción que este cometiendo, o mediante acciones activas,
  como lo puede ser tomar el control del vehículo, en caso de una emergencia.
  Para lograr que se puedan realizar efectivamente estas labores, cada uno de
  los sistemas integrados en los ITS deben de ofrecer el mejor rendimiento, es
   por esta razón que los estudios actualmente se centran en la mejora de dichos
    sistemas.\\

\section{Objetivos}

\subsection{Objetivo General}

Implementar un sistema de comunicación entre vehículo reales y virtuales, con
el fin de que se puedan realizar distintas maniobras coopertativas en distinto
 ambientes de prueba.

\subsection{Objetivos Específicos}
\begin{itemize}
	\item R
\end{itemize}

\section{Estructura del Trabajo}

Habiendo realizado la respectiva introducción al problema a tratar en el
proyecto, así como los objetivos de este, a continuación se presentarán 7
capítulos más con sus respectivos resúmenes.\\

\par En el capítulo 2, se presentará una descripición del instituto receptor,
TECNALIA \textit{Research \& Innovation}, exponiendo su historia y una breve
 presentación del equipo de Automated Driving, con el cual se realizó este
  proyecto, dando a conocer sus objetivos y los vehículos que cuenta.\\

% file thesis.tex
% Archivo thesis.tex
% Documento maestro que incluye todos los paquetes necesarios para el documento
% principal.



\documentclass[oneside,12pt,letterpaper]{report}
\tolerance=1000
\hbadness=10000
\raggedbottom

% Paquetes para manejar graficos
\usepackage{float}
\usepackage{epsf}
\usepackage[pdftex]{graphicx}
\usepackage{graphicx}
\usepackage{epsfig}
\usepackage{caption}
\usepackage{mathpazo}
% Simbolos matematicos
\usepackage{latexsym,amssymb}
% Paquetes para presentar una tesis decente.
\usepackage{setspace,cite} % Doble espacio para texto, espacio singular para
                           % los caption y pie de pagina

% Paquetes no utilizados para citas
\usepackage{booktabs}
\usepackage{adjustbox}

%figuras en modo cuadrado con texto a los lados
\usepackage{wrapfig}
\usepackage{alltt}

% Acentos
\usepackage[spanish,activeacute]{babel} %\usepackage[spanish,activeacute,es-lcroman]{babel}
\usepackage[spanish]{translator}
\usepackage[utf8]{inputenc}
\usepackage{color, xcolor, colortbl}
\usepackage{multirow}
\usepackage{subfig}
\usepackage[OT1]{fontenc}
\usepackage{tocbibind}
\usepackage{anysize}
\usepackage{listings}
\usepackage{enumerate}
% Opciones para los glosarios
\usepackage{url}
\usepackage{amsthm}
\usepackage{amsmath}
\usepackage{fancyhdr} % Necesario para los encabezados
\usepackage{fancyvrb}
\usepackage{makeidx} % En caso de necesitar indices.
\usepackage{wrapfig}
%Colocar epígrafos al principio del capítulo
\usepackage{epigraph}
%para el espacio de indentacion
\usepackage{enumitem}
\usepackage{pdfpages}
%paquete de prueba para apendices
\usepackage[toc,page]{appendix}
%Paquete para colocar los titulos centrados
\usepackage{titlesec}
\titleformat{\chapter}[display]
{\normalfont\huge\bfseries\centering}{\chaptertitlename\ \thechapter}{20pt}{\Huge}


\makeindex  % Necesitado para los indices

% Definiciones para definicions, teoremas y lemas
\theoremstyle{definition} \newtheorem{definicion}{Definici\'{o}n}
\theoremstyle{plain} \newtheorem{teorema}{Teorema}
\theoremstyle{plain} \newtheorem{lema}{Lema}

% Para la creacion de los pdfs
\usepackage{hyperref}

% Para resolver el lio del Unicode para la informacion de los PDFs
% En pdftitle coloca el nombre de su proyecto de grado/pasantia.
% En pdfauthor coloca su nombre.
\hypersetup{
    pdftitle = {Optimización de un medidor eléctrico inteligente bifásico},
    pdfauthor={Andres Suarez Figueroa},
		pdfsubject={Libro de tésis},
    pdfkeywords={},
    colorlinks,
    citecolor=black,
    filecolor=black,
    linkcolor=black,
    urlcolor=black,
    backref,
    pdftex
}
%incluir el glosario
\usepackage[toc,acronym]{glossaries}

% Crea el glosario
\makeglossaries

%entradas a los acronimos... hay que hacer una compilacion especial que no conozco
%%%%%%%%%%%%%%%%%%%%%%%%%%%%%%%%%%%%%%%%%%%%%%%%%%%%%%%%%%%%%%%%%%%%%%%%%%%%%%%%%%%%%%%%%%%%%%%%%%%%%
%\newacronym{lvm}{LVM}{Logical Volume Manager}

%\newacronym{svm}{SVM}{support vector machine}
%%%%%%%%%%%%%%%%%%%%%%%%%%%%%%%%%%%%%%%%%%%%%%%%%%%%%%%%%%%%%%%%%%%%%%%%%%%%%%%%%%%%%%%%%%%%%%%%%%%%%

% Incluye el glosario
%\input{glossary.tex}

% Para crear la hoja escaneada de las firmas
\usepackage[absolute]{textpos}

% Pone los nombres y las opciones para mostrar los codigos fuentes
\lstset{language=C, breaklines=true, frame=single, showstringspaces=false,
        showtabs=false, numbers=left, keywordstyle=\color{black},
        basicstyle=\footnotesize, captionpos=b }
\renewcommand{\lstlistingname}{C\'{o}digo fuente}
\renewcommand{\lstlistlistingname}{\'{I}ndice de c\'{o}digos fuentes}

% Dimensiones de la pagina
\setlength{\headheight}{15pt}
\marginsize{3cm}{2cm}{2cm}{2cm}

% Se pueden omitir para que no compile ciertos capitulos.
\includeonly{header, intro, ssimilar, herramienta, resultados, conclusiones}

%\selectspanish*

%%%%%%%%%%%%%%%%%%%%%%%%%%%%%%%%%%%%%%%%%%%%%%%%%%%%%%%%%%%%%%%%%%%%%%%%%%%
%%%%%%%%%%%%%%%%      end of preamble and start of document     %%%%%%%%%%%
%%%%%%%%%%%%%%%%%%%%%%%%%%%%%%%%%%%%%%%%%%%%%%%%%%%%%%%%%%%%%%%%%%%%%%%%%%%
\usepackage{subfig}

\begin{document}

\pagenumbering{Roman}
% Carátula
%% ***************************************************
% CARÁTULA DEL TRABAJO DE PASANTÍA
% ***************************************************
% \begin{titlepage}
\thispagestyle{empty}
    \begin{figure}[!ht]
        \begin{center}
            \includegraphics[scale=0.3]{../Imagenes/cebolla.jpg}
        \end{center}
    \end{figure}

    \vspace{-1.cm}

    \begin{center}
		\begin{large}
			UNIVERSIDAD SIMÓN BOLÍVAR\\
		\end{large}
    		\textbf{DECANATO DE ESTUDIOS PROFESIONALES}\\
	  		\textbf{COORDINACIÓN DE INGENIERÍA ELECTRÓNICA}\\

    \end{center}


    %\begin{center}
        %TRABAJO DE GRADO\\
    %\end{center}

    \vspace{4cm}
    \begin{center}
    %\large {
    \textbf{OPTIMIZACIÓN DE UN MEDIDOR ELÉCTRICO INTELIGENTE BIFÁSICO}\\
    %}
    \end{center}

    \vspace{3cm}
    \begin{center}
        Por:\\
        Andrés Suárez Figueroa
    \end{center}

		    \vspace{1cm}
    \begin{center}
        \textbf{PROYECTO DE GRADO}\\
				Presentado ante la Ilustre Universidad Simón Bolívar\\
				como requisito parcial para optar al Título de\\
				Ingeniero Electrónico\\
    \end{center}

    %\vspace{\stretch{1}}
	\vspace{1cm}
    \begin{center}
		\textbf{Sartenejas, en algún momento de 2018}
    \end{center}
    \pagebreak
% \end{titlepage}


%Página de título
%\input{../Portadas/Portada2.tex}

% Pagina de acta final (vacio)
%\input{acta.tex}

\setcounter{secnumdepth}{3}
\setcounter{tocdepth}{4}

% Define encabezado numeros romanos y como se separan los captiulos y las
% secciones
\addtolength{\headheight}{3pt}
\pagestyle{fancyplain}

\renewcommand{\chaptermark}[1]{\markboth{\chaptername\ \thechapter:\,\ #1}{}}
\renewcommand{\sectionmark}[1]{\markright{\thesection\,\ #1}}

\onehalfspacing

\lhead{}
\chead{}
\rhead{}
\renewcommand{\headrulewidth}{0.0pt}
\lfoot{}
\cfoot{\fancyplain{}{\thepage}}
\rfoot{}


% Pagina de resumen
%% ***************************************************
%   Resumen del trabajo de investigacion
% ***************************************************
% \begin{titlepage}
%\thispagestyle{empty}
\begin{figure}[ht]
    \begin{center}
        \includegraphics[scale=0.25]{../Imagenes/cebolla.jpg}
    \end{center}
\end{figure}

\vspace{-1.cm}
\addcontentsline{toc}{chapter}{Resumen}
     \begin{center}
		\begin{small}
			UNIVERSIDAD SIMÓN BOLÍVAR\\
    		DECANATO DE ESTUDIOS PROFESIONALES\\
	  		COORDINACIÓN DE INGENIERÍA ELECTRÓNICA\\
		\end{small}
    \end{center}

    \vspace{-0.4cm}
    \begin{center}
    \small {
    \textbf{OPTIMIZACIÓN DE UN MEDIDOR ELÉCTRICO INTELIGENTE BIFÁSICO}\\
    }
    \end{center}

		\vspace{-0.4cm}
    \begin{center}
    \small {
    PROYECTO DE GRADO,\\
		PRESENTADO POR:\\
Andrés Suárez Figueroa, Carnet 12-10925\\
    }
    \end{center}


		\begin{center}
    %\small {
    \textbf{RESUMEN}\\
    %}
    \end{center}

\noindent
%
Los C-ITS (del inglés, \textit{Cooperative Intelligent Transportation Systems}) son cada vez más, una realidad en la sociedad, aportando soluciones, y comodidades a la hora de manejar, resuelven una gran cantidad de inconvenientes presentes en el ámbito automovilístico. Con la finalidad de contribuir con el desarrollo de los mismos, el presente trabajo busca elaborar algoritmos inteligentes para realizar distintas maniobras cooperativas entre vehículos automatizados y semi-automatizados, basados en comunicaciones V2V, los cuales se puedan ver implementados en entornos reales y virtuales. Dentro de estas maniobras ejecutadas se pueden destacar: El ACC (del inglés, \textit{Adaptative Cruise Control}), \textit{Stop and Go} y ACC con Control Lateral, las cuales se diseñaron bajo la lógica difusa y se probaron empleando el simulador Dynacar. Simulador, que, conjunto a los vehículos Renault Twizy sirvieron para validar, no solo las maniobras sino también el sistema de comunicación comercial, al probar los mismos en tres entornos distintos, PC - PC, Vehículo - Vehículo (V2V), y Vehículo - PC.

\vspace{0.5cm}
\noindent
\textbf{Palabras Claves: Sistemas Inteligentes de Transporte Cooperrativos, Maniobras Cooperativas, Comunicaciones V2V, ACC, Control Lateral, Sistema de Comunicación Comercial}
    \\
\pagebreak
%% \end{titlepage}


% Pagina de agradecimientos (opcional)
%%%Dedicatoria!!!!!!!!!!!!!!!!!!!!!!
\pagebreak
\thispagestyle{empty}
\begingroup
%\rightskip0.5cm
\vspace*{1,5cm}
\begin{center}
\LARGE\textbf{Agradecimientos}
\end{center}
\small
\vspace*{2cm}



\textit{A mis padres, Maria Emilia y Rubén Dario, a quienes debo mis logros. Gracias por darme todo, y mucho más.}\\

\textit{A mis hermanos, Rubén Alejandro y Andrea Carolina, quienes me han apoyado y ayudado en cada decisión de mi vida.}\\

\textit{A mis amigos, que me han acompañado durante toda esta travesía que ha sido la universidad, haciéndola más divertida y motivadora.}\\

\textit{Al profesor Gerardo Fernández, que al igual que Joshué Pérez, representan un modelo a seguir, tanto personal como profesionalmente. Gracias por hacer realidad este proyecto.}\\

\textit{Al equipo de Automated Driving, por apoyarme y guiarme a lo largo de este proyecto.}\\

\textit{A Dios, por bendecirme en cada día, con todas estas personas que me ayudan y me dan fuerzas para superar exitosamente cada momento de mi vida.}\\


\endgroup
%\hfill\textit{lo esencial es invisible a la vista, pero visible }

%\hfill\textit{con el coraz\'on; ustedes, principitos, me han domesticado.}


\pagebreak

%%Lista de acronimos (no aparece como un capitulo)
\chapter*{Lista de Abreviaturas} %No contarlo como capitulo del trabajo
\addcontentsline{toc}{chapter}{Lista de Abreviaturas} %Incluirlo en el indice
%\thispagestyle{empty} %Sin numero de pagina u otros añadidos en margenes

\begin{tabular}{p{2cm} p{13.3cm}}%{>{\centering}m{2.45cm}>{\raggedright}m{11.5cm}>{\centering\arraybackslash}m{2cm}}%
ACC & Control crucero adaptativo, del inglés: \textit{Adaptative Cruise Control}.\\
ADAS & Sistemas avanzados de asistencia al conductor, del inglés: \textit{Advanced Driver Assistance Systems}.\\
AP & puntos de acceso, del inglés: \textit{Access Point}.\\
ARIB & Asociación de Industrias y Negocios de Radio, del inglés: \textit{Association of Radio Industries and Businesses}.\\
ASK & Modulación por Desplazamiento de Amplitud, del inglés: \textit{Amplitud-Shift Keying}.\\
CACC & Control de Crusero Adaptativo Cooperativo, del inglés: \textit{Cooperative Adaptative Cruise Control}.\\
CACS & Sistema Integral Automovilístico de Control de Tráfico, del inglés: \textit{Comprehensive Automovile Traffic Control System}.\\
CALM & Arquitectura de Acceso a Comunicaciones Terrestres Móviles, del inglés: \textit{Communications Acces for Land Mobiles}.\\
CAM & Mensaje de Conciencia Cooperativo (del inglés: \textit{Cooperative Awareness Message}.\\
CAN & Red Controladora del Área, del inglés: \textit{Controller Area Network}.\\
CCH & Canal de Control, del inlgés: \textit{Control CHanel}.\\
CEN & Comité Europeo de Estandarización, del inglés: \textit{European Committee for Standardization}.\\
C-ITS & Sistemas Inteligentes de Transporte Cooperativos, del inglés: \textit{Cooperative Intelligent Transport Systems}.\\
CSMA/CA & Acceso Múltiple por Detección de Portadora y Prevención de Colisiones, del inglés: \textit{Carrier Sense Multiple Acces with Collision Avoidance}.\\

\end{tabular}

\pagebreak
\begin{tabular}{p{2cm} p{13.3cm}}
DGPS & Sistema de Posicionamiento Global Diferencial, del inglés \textit{Differential Global Positioning System}.\\
DENM & Mensaje de Notificación Decentralizada del Ambiente, del inglés: \textit{Decentralized Environmental Notification Message}.\\
DLC & Control de Enlace de Datos, del inglés: \textit{Data Link Level}.\\
DSRC & Comunicaciones de Corto Alcance, del inglés: \textit{ Dedicated Short Range Communication}.\\
DSS & Espectro Expandido por Secuencia Diercta, del inglés: \textit{Direct Sequence Spread Spectrum}.\\
EAP & Protocolo de Seguridad Extendida, del inglés: \textit{Extensible Authentiticatiion Protocol}.\\ 
EDCA & Acceso Mejorado al Canal Distribuido, del inglés: \textit{Enhanced Distributed Channel Access}.\\
ECU & Unidad de Control del Motor, del inglés: \textit{Engine Control Unit}.\\
ESO & Organización Europea de Estandarización, del inglés: \textit{European Standardization Organization}.\\
ETSI & Instituto Europeo de Normas de Telecomunicaciones, del inglés: \textit{ European Telecommunications Standars Institute}.\\
FCC & Comisión Federal de Comunicaciones de Estados Unidos, del inglés: \textit{Federal Communications Commision}.\\
FHSS & Espectro Expandido por Salto de Frecuencia, del inglés: \textit{ Frequency Hopping Spread Spectrum}.\\
GPS & Sistemas de posicionamiento global, del inglés: \textit{Global Positioning Systems}.\\
GUI & Interfaces Gráficas de Usuario, del inglés: \textit{Graphical User Interface}.\\
IDE & Entorno de Desarrollo Integrado, del inglés: \textit{Integrated Development Environment}.\\
IEEE & Instituto de ingenieros electricistas y electrónicos, del inglés: \textit{Institute of Electrical and Electronics Engineers}.\\
IMU & Unidad de Medición Inercial, del inglés: \textit{Intertial Measurement Unit}.\\
ISM & Industriales, Científicas y Médicas, del inglés: \textit{Industrial, Scientific and Medical}.\\

\end{tabular}

\pagebreak
\begin{tabular}{p{2cm} p{13.3cm}}
ISO & Organización Internacional de Normalización, del inglés: \textit{International Organization for Standardization}.\\
ITS & Sistemas inteligentes de transporte, del inglés: \textit{Intelligent Transportation Systems}.\\
ITU & Internacional de Telecomunicaciones, del inglés: \textit{International Telecommunications Union}.\\
LIDAR & Detección de Luz y Medición de Distancia, del inglés: \textit{Light Detection And Ranging}.\\
MAC & Control de Acceso al Medio, del inglés: \textit{Media Acces Control}.\\
MANET & Red Ad-Hoc Móvil, del inglés: \textit{Moblie Ad-Hoc Network}.\\
MCO & Operador Multi Canal, del inglés: \textit{Multi-Channel Operator}.\\
MITTI & Ministerio de Industria y Comercio Internacional de Japón, del inglés: \textit{Ministry of International Trade and Industry}.\\
OCB & Esquema de Encriptación Autenticada, del inglés: \textit{ Offset Codebook Mode}.\\
OBU & Unidad a Bordo, del inglés: \textit{On Board Unti}.\\
OFDM & Modulación por División Ortogonal de Frecuencias, del inglés: \textit{Orthogonal Frequency Division Multiplexing}.\\
ONU & Organización de las Naciones Unidas del inglés: \textit{Organization of United Nation}.\\
OSI & Interconexión de Sistemas Abiertos, del inglés: \textit{Open System Interconnection}.\\
PLC & Controlador Lógico Programable, del inglés: \textit{Programable Logic Contollers}.\\
PSK & Modulación por Desplazamiento de Fase, del inglés: \textit{Phase Shift Keying}.\\
PWM & Modulación por Ancho de Banda, del inglés: \textit{Pulse-Width Modulation}.\\
QoS & Calidad de Servicio, del inglés: \textit{Quality of Service}.\\
RSU & Unidades en Vía, del inglés: \textit{ Road Side Unit's}.\\
SAE & Sociedad de Ingenieros de Automoción, del inglés:\textit{ Society of Automotive Engineers}.\\
TCP & Protocolo de Control de Transmisión, del inglés: \textit{Transmission Control Protocol}.\\
\end{tabular}

\pagebreak
\begin{tabular}{p{2cm} p{13.3cm}}
UDP & Protocolo de Datagrama de Usuario, del inglés: \textit{User Datagram Protocol}.\\
V2I & Vehículo con Infraestructura, del inglés: \textit{Vehícle to Infraestructure}.\\
V2N & Vehículo con la Red, del inglés: \textit{Vehícle to Network}.\\
V2P & Vehículo con Peatón, del inglés: \textit{Vehícle to Pedestrian}.\\
V2V & Vehículo con Vehículo, del inglés: \textit{Vehícle to Vehícle}.\\
V2X & Vehículo con Todo, del inglés: \textit{Vehícle to Everething}.\\
VANET & Redes Ad-Hoc Vehicular, del inglés: \textit{Vehicular Ad-Hoc Network}.
\end{tabular}

% Pagina de dedicatoria (opcional)
%%%Dedicatoria!!!!!!!!!!!!!!!!!!!!!!
\pagebreak
\thispagestyle{empty}
\begingroup
\rightskip0.5cm
\small
\vspace*{7,5cm}



\hfill\textit{A mi familia}\hspace{20pt}


\hfill\textit{Por su guía y apoyo a lo largo de mi vida}\hspace{20pt}


\hfill\textit{}\hspace{35pt}
\endgroup


\pagebreak


\pagebreak
\thispagestyle{empty}


% Crea la tabla de contenidos
\tableofcontents

% Crea la lista de figuras
\listoffigures

% Crea la lista de cuadros
\renewcommand{\listtablename}{Índice de tablas}\listoftables

\clearpage

% Define encabezado en numeros arabicos
\pagenumbering{arabic}

\fancyhf{} % Redefine el encabezado
\lhead{}
\chead{}
\rhead{\fancyplain{}{\thepage}}
\renewcommand{\headrulewidth}{0.0pt}
\lfoot{}
\cfoot{}
\rfoot{}

\doublespacing


% Incluye los archivos deseados - El contenido de
%su proyecto de grado.
%%Intro
\chapter{Introducción}\label{sec:capitulo1}
\thispagestyle{empty}

\begingroup
\rightskip0.5cm
\small

\endgroup

\section{Antecedentes}
En la actualidad la evolución de las ciudades inteligentes se ha vuelto un punto
de interés en la mayoría de los países. La integración de las nuevas tecnologías
que permiten medir variables que antes eran inimaginables han dado paso a la
creación de productos que son capaces de controlar sistemas que mejoran la
condición de vida de los ciudadanos y el medio ambiente. En vías de cuidar el
medio ambiente y una mejor calidad de vida se han comenzado a desarrollar
las Smart Grids, redes de generación distribuidas inteligentes que poseen la
capacidad de medir el flujo de potencia de manera bidireccional, característica
que antes no existía, y al mismo tiempo permiten controlar de una forma más
eficiente y limpia la distribución de energía. Para este fin, es necesario dotar
a las redes de distribución de inteligencia con medidores inteligentes, capaces
de dotar con precisión los datos del flujo de potencia en ambos sentidos de la
red en tiempo real.\\

\par Un claro ejemplo de esta cualidad, es la invención del automóvil, el cual
 llegó como solución al problema del transporte por tierra, de una forma más
 cómoda y práctica. Dicho problema no se resolvió hasta que el ingeniero alemán
 Karl Friedrich Benz creó el primer automóvil en 1885 \cite{cernuschi2005cuatro},
 abriendo de esta forma, las puertas a un nuevo mundo para la investigación.\\


\section{Justificación y Planteamiento del Problema}

En la actualidad, más de 1,25 millones de presonas mueren cada año como
consecuencia de accidentes de tránsito, y aproximadamente 50 millones
 sufren traumatismos no mortales, los cuales pueden llegar a producir
 alguna discapacidad \footnote{http://www.who.int/mediacentre/factsheets/fs358/es/}.
  Dichos siniestros son causados, en su mayoría, por la imprudencia del ser
   humano. Si no se toman medidas correctivas se espera que estas cifras tan
   alarmantes aumenten para el año 2030, de tal forma que se conviertan la
   séptima causa de muerte en el mundo. Es por eso que la organización de las
   naciones unidas, ONU (del inglés, \textit{Organization of United Nation})
   adoptó \textit{La agenda 2030} para el desarrollo sostenible, donde se
   espera que para el 2020 se disminuyan estos números a la mitad, a través de
   dsitintos planes.\\

\par Para apoyar estas soluciones que se pretenden poner en práctica, el
 desarrollo e implementación de los ITS juegan un papel muy importante,
 ya que los mismos buscan solventar los fallos del ser humano, ya sea mediante
  acciones pasivas, como lo puede ser una simple notificación al conductor de
  alguna falla o infracción que este cometiendo, o mediante acciones activas,
  como lo puede ser tomar el control del vehículo, en caso de una emergencia.
  Para lograr que se puedan realizar efectivamente estas labores, cada uno de
  los sistemas integrados en los ITS deben de ofrecer el mejor rendimiento, es
   por esta razón que los estudios actualmente se centran en la mejora de dichos
    sistemas.\\

\section{Objetivos}

\subsection{Objetivo General}

Implementar un sistema de comunicación entre vehículo reales y virtuales, con
el fin de que se puedan realizar distintas maniobras coopertativas en distinto
 ambientes de prueba.

\subsection{Objetivos Específicos}
\begin{itemize}
	\item R
\end{itemize}

\section{Estructura del Trabajo}

Habiendo realizado la respectiva introducción al problema a tratar en el
proyecto, así como los objetivos de este, a continuación se presentarán 7
capítulos más con sus respectivos resúmenes.\\

\par En el capítulo 2, se presentará una descripición del instituto receptor,
TECNALIA \textit{Research \& Innovation}, exponiendo su historia y una breve
 presentación del equipo de Automated Driving, con el cual se realizó este
  proyecto, dando a conocer sus objetivos y los vehículos que cuenta.\\

\input{../Capitulos/Marco_Teorico.tex}
\chapter{Desarrollo}\label{sec:capitulo3}
\thispagestyle{empty}

\begingroup
\rightskip0.5cm
\small

\endgroup

\section{Estableciendo parametros de muestreo}

\section{Eleccion de dispositivo para procesamiento}
\par Se evaluaron unidades de procesamiento de distintas compañias y los
disponibles en el laboratorio del grupo de Sistemas Industriales y Electrónica
de Potencia capaces de realizar las funciones del MEI. Por parte de la empresa
Texas Instruments los dispositivos TMS320F28027 y TMS320F28069 pertenecientes
a la familia C2000 - Piccolo, de la empresa National Instruments el myRIO-1900
y por último de la familia Intel el Arduino UNO.

Las caracteristicas evaluadas de cada microcontrolador son las siguientes:
\begin{\begin{itemize}
  \item Velocidad de adquisición
  \item Resolución de ADC
  \item Velocidad de procesamiento
  \item Conectividad
  \item Acceso directo a memoria
\end{itemize}}

\begin{table}[]
\centering
\caption{My caption}
\label{my-label}
\begin{tabular}{ccccc}
\hline
\textbf{Dispositivo}       & \textbf{Arduino Uno} & \textbf{myRIO-1900} & \textbf{TMS320F28027}      & \textbf{TMS320F28069} \\ \hline
Velocidad de adquisición   & 8950 Hz              & 40 KHz              & 3.4 MHz                    & 4.6 MHz               \\
Resolución ADC             & 10                   & 12                  & 12 Bits                    & 12 Bits               \\
Velocidad de procesamiento & 16 MHz               & 667 MHz             & \multicolumn{1}{r}{60 MHz} & 90 MHz                \\
Conectividad               & Serial               & Serial / Wifi       & Serial                     & Serial                \\
Acceso directo a memoria   &                      &                     &                            &                       \\ \hline
\end{tabular}
\end{table}

\chapter{Resultados}\label{sec:Resultados}
\thispagestyle{empty}

\begingroup
\rightskip0.5cm
\small

\endgroup

\section{Resultados}

\chapter{Conclusiones y recomendaciones}\label{sec:Conclusiones_recomendaciones}
\thispagestyle{empty}

\begingroup
\rightskip0.5cm
\small

\endgroup

\section{Conclusiones}

% Crea el glosario
%\printglossary%[title={Acrónimos},toctitle={Acrónimos}]

% Establece las citas y bibliografia
%\bibliographystyle{unsrt}
\bibliography{LibroRef}

%\renewcommand{\appendixname}{Apéndice} %Lo que aparecerá en el titulo donde se ubica el apendice
%\renewcommand{\appendixtocname}{Apéndices}%lo que aparecerá en la
%\renewcommand{\appendixpagename}{Apéndices}

%\addto{\captionsenglish}{\renewcommand{\appendix}{MyAppx}}
%\appendix
%\appendixpage%{Apéndice A}
%\clearpage

%\appendixpage
%\chapter{Publicación: ``Optimal Energy Consumption Algorithm based on Speed Reference Generation for Urban Environments''. En revisión}
%\includepdf[pages={1-9}]{Apendices/SpeedControl_v02_revista.pdf}

%\chapter{Documentación del módulo ``ProfileOptimizer''}
%\includepdf[pages={1-4}]{Apendices/RTMaps4ProfileOptimizer.pdf}

\end{document}

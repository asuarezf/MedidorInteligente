% ***************************************************
%   Resumen del trabajo de investigacion
% ***************************************************
% \begin{titlepage}
%\thispagestyle{empty}
\begin{figure}[ht]
    \begin{center}
        \includegraphics[scale=0.25]{../Imagenes/cebolla.jpg}
    \end{center}
\end{figure}

\vspace{-1.cm}
\addcontentsline{toc}{chapter}{Resumen}
     \begin{center}
		\begin{small}
			UNIVERSIDAD SIMÓN BOLÍVAR\\
    		DECANATO DE ESTUDIOS PROFESIONALES\\
	  		COORDINACIÓN DE INGENIERÍA ELECTRÓNICA\\
		\end{small}
    \end{center}

    \vspace{-0.4cm}
    \begin{center}
    \small {
    \textbf{OPTIMIZACIÓN DE UN MEDIDOR ELÉCTRICO INTELIGENTE BIFÁSICO}\\
    }
    \end{center}

		\vspace{-0.4cm}
    \begin{center}
    \small {
    PROYECTO DE GRADO,\\
		PRESENTADO POR:\\
Andrés Suárez Figueroa, Carnet 12-10925\\
    }
    \end{center}


		\begin{center}
    %\small {
    \textbf{RESUMEN}\\
    %}
    \end{center}

\noindent
%
Los C-ITS (del inglés, \textit{Cooperative Intelligent Transportation Systems}) son cada vez más, una realidad en la sociedad, aportando soluciones, y comodidades a la hora de manejar, resuelven una gran cantidad de inconvenientes presentes en el ámbito automovilístico. Con la finalidad de contribuir con el desarrollo de los mismos, el presente trabajo busca elaborar algoritmos inteligentes para realizar distintas maniobras cooperativas entre vehículos automatizados y semi-automatizados, basados en comunicaciones V2V, los cuales se puedan ver implementados en entornos reales y virtuales. Dentro de estas maniobras ejecutadas se pueden destacar: El ACC (del inglés, \textit{Adaptative Cruise Control}), \textit{Stop and Go} y ACC con Control Lateral, las cuales se diseñaron bajo la lógica difusa y se probaron empleando el simulador Dynacar. Simulador, que, conjunto a los vehículos Renault Twizy sirvieron para validar, no solo las maniobras sino también el sistema de comunicación comercial, al probar los mismos en tres entornos distintos, PC - PC, Vehículo - Vehículo (V2V), y Vehículo - PC.

\vspace{0.5cm}
\noindent
\textbf{Palabras Claves: Sistemas Inteligentes de Transporte Cooperrativos, Maniobras Cooperativas, Comunicaciones V2V, ACC, Control Lateral, Sistema de Comunicación Comercial}
    \\
\pagebreak
%% \end{titlepage}

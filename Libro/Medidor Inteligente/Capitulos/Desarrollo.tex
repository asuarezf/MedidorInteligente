\chapter{Desarrollo}\label{sec:capitulo3}
\thispagestyle{empty}

\begingroup
\rightskip0.5cm
\small

\endgroup




\section{Estableciendo parámetros a medir}%-------------------------------------
\par Como ya se mencionaron antes los estandares internacionales establecen:

\begin{itemize}
  \item Valores nominales de la red eléctrica
  \item Espcificaciones de los medidores inteligentes
    \begin{itemize}
      \item Eléctricas
      \item Medición (BUSCAR OTRO TERMINO)
    \end{itemize}
\end{itemize}

\subsection{Eléctricas}
\par La norma EDC E-354-1608 establece los valores nominales de la red eléctrica,
los cuales deben ser medidos por el medidor inteligente.

\begin{table}[H]
\centering
\caption{\label{Valores_nominales_re}Valores nominales de operacion de la red eléctrica, extracto norma E-354-D-1608}
\resizebox{\textwidth}{!}{
\begin{tabular}{cccc}
\hline
\textbf{\begin{tabular}[c]{@{}c@{}}Característica\\ Medidor\end{tabular}} & \multicolumn{3}{c}{\textbf{Medición Eléctrica}} \\ \hline
Tipo & 1F, 2 Hilos & 1F, 3 Hilos & 3F, 4 Hilos \\
Rango de intensidad (A) & 10-60 & 15-80 & 15-100 \\
Tensión nominal (V) & 120 & \begin{tabular}[c]{@{}c@{}}2 x 120/240\\ 2 x 240/480\end{tabular} & \begin{tabular}[c]{@{}c@{}}3 x 208Y/120\\ 3 x 240/2 x 120\\ 3 x 416Y/240\\ 3 x 480Y/277\end{tabular} \\
Rigidez dielectrica?? (KV) & 4 & 4 & 4 \\
Frecuencia (Hz) & 60 & 60 & 60 \\ \hline
\end{tabular}}
\end{table}

\par La tabla \ref{Valores_nominales_re} es un estracto de la norma ya mencionada en
 de la cual determinamos los valores en el cual operan los medidores inteligentes;
 frecuencia de 60 Hz, 120 V nominales y corriente entre 10-60 A.

\subsection{Medición}

La norma IEC 61000-4-30 especifica que los medidores de clase B deben ser capaces
 de medir hasta el 30avo armónico.

\textbf{NORMA QUE ESTABLECE LOS ERRORES PORCENTUALES PERMITIDOS}

\section{Tarjeta de acondicionamiento}%-----------------------------------------




\section{Eleccion de dispositivo para procesamiento}%----------------------------
\par Se evaluaron unidades de procesamiento de distintas compañias y los
disponibles en el laboratorio del grupo de Sistemas Industriales y Electrónica
de Potencia capaces de realizar las funciones del MEI. Por parte de la empresa
Texas Instruments los dispositivos TMS320F28027 y TMS320F28069 pertenecientes
a la familia C2000 - Piccolo, de la empresa National Instruments el myRIO-1900
y por último de la familia Intel el Arduino UNO.

Las caracteristicas evaluadas de cada microcontrolador son las siguientes:
\begin{itemize}
  \item Velocidad de adquisición
  \item Resolución de ADC
  \item Velocidad de procesamiento
  \item Conectividad
  \item Acceso directo a memoria
\end{itemize}

\begin{table}[H]
\centering
\caption{\label{Especificaciones_dispositivos_comerciales_evaluados}Especificaciones de los dispositivos comerciales evaluados}
\resizebox{\textwidth}{!}{\begin{tabular}{ccccc}
\hline
\textbf{Dispositivo} & \textbf{Arduino Uno} & \textbf{myRIO-1900} & \textbf{TMS320F28027} & \textbf{TMS320F28069} \\ \hline
\begin{tabular}[c]{@{}c@{}}Frecuencia de \\ muestreo\end{tabular} & 8950 Hz & 40 KHz & 3.4 MHz & 4.6 MHz \\
\begin{tabular}[c]{@{}c@{}}Velocidad de\\ procesamiento\end{tabular} & 16 MHz & 667 MHz & 60 MHz & 90 MHz \\
Conectividad & Serial & Serial / Wifi & Serial & Serial \\
\begin{tabular}[c]{@{}c@{}}Acceso directo \\ a memoria\end{tabular} & No & Si & Si & Si \\ \hline
\end{tabular}}
\end{table}

\subsection{Frecuencia de muestreo (MODIFICAR ARMONICO)}
\par Los estandares internaciones (averiguar que estandar) de la IEC establecen
que los medidores inteligentes de clase B, los utilizados en las zonas residenciales,
deben detectar al menos hasta el 51avo armónico. La frecuencia de la señal a medir es de 60 Hz, la
correspondiente a la red eléctrica en Venezuela, ubicando el 51avo armónico a una frecuencia de
3060 Hz. Según el teorema de Nyquist se requiere una frecuencia de muestreo de al menos el doble
de dicha frecuencia, 6080 Hz.
\par Los ADC de todos los dispositivos se encuentran multiplexados, lo que implica que
la frecuencia de muestreo se dividira por el numero de señales a obtener. En este caso
se utilizaran dos canales, en uno se obtendra la señal de voltaje y en el otro la de
corriente. Duplicando asi la frecuencia de muestreo necesaria para muestrear la señal,
12.16 KHz.

\subsection{Velocidad de procesamiento y acceso directo a memoria}
\par Al ser un sistema embebido y en tiempo real, es necesario que el dispositivo
sea capaz de realizar grandes cantidades de instrucciones y ejecuciones en paralelo.
Por otro lado los dispositivos que posea acceso directo a memoria son mas capaces de
realizar operaciones debido a que este acceso es capaz de trabajar en paralelo con
el procesamiento de los datos (ARREGLAR).

\subsection{Comunicación}
\par Siguiendo los estandares (averiguar estandar) para la comunicacion de los
medidores inteligentes, se recomienda que el dispostivo sea capaz de conectarse a una
red y enviar datos mediante el protocolo TCP/IP. \\

\par Teniendo en cuenta todo esto tomamos como parametro mas importante la frecuencia de
muestreo, descartando de este modo al dispositivo Arduino Uno que solo posee una
máxima frecuencia de 8950 Hz con un canal. Los dispositivos restantes cumplen con
sosobra la frecuencia de muestreo minima necesaria, teniendo que tomar en cuenta los
siguientes aspectos. La velocidad de proceseamiento en los tres dispositivos es
sumamente elevada y ademas de esto cuentan con la capacidad de acceso directo a
memoria dando un uso mas optimizado de sus recursos. Sin embargo por la parte de
comunicacion los dispositivos de Texas Instruments requieren de modulos extras
para obtener conexión WIFI y poder transmitir los datos por el protocolo TCP/IP,
dejando asi en ventaja al dispositivo myRIO. De tal modo se eligio el dispositivo
myRIO ya que cumple con todas las caracteristicas necesarias y este ya se encuentra
disponible en los laboratorios de la agrupación al contrario de los TI que es necesario
realizar su compra.

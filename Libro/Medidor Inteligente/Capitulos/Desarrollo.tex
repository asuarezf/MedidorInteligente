\chapter{Desarrollo}\label{sec:capitulo3}
\thispagestyle{empty}

\begingroup
\rightskip0.5cm
\small

\endgroup

\section{Estableciendo parametros de muestreo}

\section{Eleccion de dispositivo para procesamiento}
\par Se evaluaron unidades de procesamiento de distintas compañias y los
disponibles en el laboratorio del grupo de Sistemas Industriales y Electrónica
de Potencia capaces de realizar las funciones del MEI. Por parte de la empresa
Texas Instruments los dispositivos TMS320F28027 y TMS320F28069 pertenecientes
a la familia C2000 - Piccolo, de la empresa National Instruments el myRIO-1900
y por último de la familia Intel el Arduino UNO.

Las caracteristicas evaluadas de cada microcontrolador son las siguientes:
\begin{\begin{itemize}
  \item Velocidad de adquisición
  \item Resolución de ADC
  \item Velocidad de procesamiento
  \item Conectividad
  \item Acceso directo a memoria
\end{itemize}}

\begin{table}[]
\centering
\caption{My caption}
\label{my-label}
\begin{tabular}{ccccc}
\hline
\textbf{Dispositivo}       & \textbf{Arduino Uno} & \textbf{myRIO-1900} & \textbf{TMS320F28027}      & \textbf{TMS320F28069} \\ \hline
Velocidad de adquisición   & 8950 Hz              & 40 KHz              & 3.4 MHz                    & 4.6 MHz               \\
Resolución ADC             & 10                   & 12                  & 12 Bits                    & 12 Bits               \\
Velocidad de procesamiento & 16 MHz               & 667 MHz             & \multicolumn{1}{r}{60 MHz} & 90 MHz                \\
Conectividad               & Serial               & Serial / Wifi       & Serial                     & Serial                \\
Acceso directo a memoria   &                      &                     &                            &                       \\ \hline
\end{tabular}
\end{table}

%Lista de acronimos (no aparece como un capitulo)
\chapter*{Lista de Abreviaturas} %No contarlo como capitulo del trabajo
\addcontentsline{toc}{chapter}{Lista de Abreviaturas} %Incluirlo en el indice
%\thispagestyle{empty} %Sin numero de pagina u otros añadidos en margenes

\begin{tabular}{p{2cm} p{13.3cm}}%{>{\centering}m{2.45cm}>{\raggedright}m{11.5cm}>{\centering\arraybackslash}m{2cm}}%
ACC & Control crucero adaptativo, del inglés: \textit{Adaptative Cruise Control}.\\
ADAS & Sistemas avanzados de asistencia al conductor, del inglés: \textit{Advanced Driver Assistance Systems}.\\
AP & puntos de acceso, del inglés: \textit{Access Point}.\\
ARIB & Asociación de Industrias y Negocios de Radio, del inglés: \textit{Association of Radio Industries and Businesses}.\\
ASK & Modulación por Desplazamiento de Amplitud, del inglés: \textit{Amplitud-Shift Keying}.\\
CACC & Control de Crusero Adaptativo Cooperativo, del inglés: \textit{Cooperative Adaptative Cruise Control}.\\
CACS & Sistema Integral Automovilístico de Control de Tráfico, del inglés: \textit{Comprehensive Automovile Traffic Control System}.\\
CALM & Arquitectura de Acceso a Comunicaciones Terrestres Móviles, del inglés: \textit{Communications Acces for Land Mobiles}.\\
CAM & Mensaje de Conciencia Cooperativo (del inglés: \textit{Cooperative Awareness Message}.\\
CAN & Red Controladora del Área, del inglés: \textit{Controller Area Network}.\\
CCH & Canal de Control, del inlgés: \textit{Control CHanel}.\\
CEN & Comité Europeo de Estandarización, del inglés: \textit{European Committee for Standardization}.\\
C-ITS & Sistemas Inteligentes de Transporte Cooperativos, del inglés: \textit{Cooperative Intelligent Transport Systems}.\\
CSMA/CA & Acceso Múltiple por Detección de Portadora y Prevención de Colisiones, del inglés: \textit{Carrier Sense Multiple Acces with Collision Avoidance}.\\

\end{tabular}

\pagebreak
\begin{tabular}{p{2cm} p{13.3cm}}
DGPS & Sistema de Posicionamiento Global Diferencial, del inglés \textit{Differential Global Positioning System}.\\
DENM & Mensaje de Notificación Decentralizada del Ambiente, del inglés: \textit{Decentralized Environmental Notification Message}.\\
DLC & Control de Enlace de Datos, del inglés: \textit{Data Link Level}.\\
DSRC & Comunicaciones de Corto Alcance, del inglés: \textit{ Dedicated Short Range Communication}.\\
DSS & Espectro Expandido por Secuencia Diercta, del inglés: \textit{Direct Sequence Spread Spectrum}.\\
EAP & Protocolo de Seguridad Extendida, del inglés: \textit{Extensible Authentiticatiion Protocol}.\\ 
EDCA & Acceso Mejorado al Canal Distribuido, del inglés: \textit{Enhanced Distributed Channel Access}.\\
ECU & Unidad de Control del Motor, del inglés: \textit{Engine Control Unit}.\\
ESO & Organización Europea de Estandarización, del inglés: \textit{European Standardization Organization}.\\
ETSI & Instituto Europeo de Normas de Telecomunicaciones, del inglés: \textit{ European Telecommunications Standars Institute}.\\
FCC & Comisión Federal de Comunicaciones de Estados Unidos, del inglés: \textit{Federal Communications Commision}.\\
FHSS & Espectro Expandido por Salto de Frecuencia, del inglés: \textit{ Frequency Hopping Spread Spectrum}.\\
GPS & Sistemas de posicionamiento global, del inglés: \textit{Global Positioning Systems}.\\
GUI & Interfaces Gráficas de Usuario, del inglés: \textit{Graphical User Interface}.\\
IDE & Entorno de Desarrollo Integrado, del inglés: \textit{Integrated Development Environment}.\\
IEEE & Instituto de ingenieros electricistas y electrónicos, del inglés: \textit{Institute of Electrical and Electronics Engineers}.\\
IMU & Unidad de Medición Inercial, del inglés: \textit{Intertial Measurement Unit}.\\
ISM & Industriales, Científicas y Médicas, del inglés: \textit{Industrial, Scientific and Medical}.\\

\end{tabular}

\pagebreak
\begin{tabular}{p{2cm} p{13.3cm}}
ISO & Organización Internacional de Normalización, del inglés: \textit{International Organization for Standardization}.\\
ITS & Sistemas inteligentes de transporte, del inglés: \textit{Intelligent Transportation Systems}.\\
ITU & Internacional de Telecomunicaciones, del inglés: \textit{International Telecommunications Union}.\\
LIDAR & Detección de Luz y Medición de Distancia, del inglés: \textit{Light Detection And Ranging}.\\
MAC & Control de Acceso al Medio, del inglés: \textit{Media Acces Control}.\\
MANET & Red Ad-Hoc Móvil, del inglés: \textit{Moblie Ad-Hoc Network}.\\
MCO & Operador Multi Canal, del inglés: \textit{Multi-Channel Operator}.\\
MITTI & Ministerio de Industria y Comercio Internacional de Japón, del inglés: \textit{Ministry of International Trade and Industry}.\\
OCB & Esquema de Encriptación Autenticada, del inglés: \textit{ Offset Codebook Mode}.\\
OBU & Unidad a Bordo, del inglés: \textit{On Board Unti}.\\
OFDM & Modulación por División Ortogonal de Frecuencias, del inglés: \textit{Orthogonal Frequency Division Multiplexing}.\\
ONU & Organización de las Naciones Unidas del inglés: \textit{Organization of United Nation}.\\
OSI & Interconexión de Sistemas Abiertos, del inglés: \textit{Open System Interconnection}.\\
PLC & Controlador Lógico Programable, del inglés: \textit{Programable Logic Contollers}.\\
PSK & Modulación por Desplazamiento de Fase, del inglés: \textit{Phase Shift Keying}.\\
PWM & Modulación por Ancho de Banda, del inglés: \textit{Pulse-Width Modulation}.\\
QoS & Calidad de Servicio, del inglés: \textit{Quality of Service}.\\
RSU & Unidades en Vía, del inglés: \textit{ Road Side Unit's}.\\
SAE & Sociedad de Ingenieros de Automoción, del inglés:\textit{ Society of Automotive Engineers}.\\
TCP & Protocolo de Control de Transmisión, del inglés: \textit{Transmission Control Protocol}.\\
\end{tabular}

\pagebreak
\begin{tabular}{p{2cm} p{13.3cm}}
UDP & Protocolo de Datagrama de Usuario, del inglés: \textit{User Datagram Protocol}.\\
V2I & Vehículo con Infraestructura, del inglés: \textit{Vehícle to Infraestructure}.\\
V2N & Vehículo con la Red, del inglés: \textit{Vehícle to Network}.\\
V2P & Vehículo con Peatón, del inglés: \textit{Vehícle to Pedestrian}.\\
V2V & Vehículo con Vehículo, del inglés: \textit{Vehícle to Vehícle}.\\
V2X & Vehículo con Todo, del inglés: \textit{Vehícle to Everething}.\\
VANET & Redes Ad-Hoc Vehicular, del inglés: \textit{Vehicular Ad-Hoc Network}.
\end{tabular}
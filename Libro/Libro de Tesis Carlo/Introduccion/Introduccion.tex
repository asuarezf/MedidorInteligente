%Intro
\chapter{Introducción}\label{sec:capitulo1}
\thispagestyle{empty}

\begingroup
\rightskip0.5cm
\small

\endgroup

\section{Antecedentes}
El hombre en la búsqueda de mejorar su calidad de vida, ha logrado obtener grandes avances haciendo uso de la tecnología y así, con el pasar de los años ha podido solucionar problemas que parecían casi imposibles, mejorar los inventos que se pensaban como óptimos, y crear lo que una vez parecía inimaginable.\\ 

\par Un claro ejemplo de esta cualidad, es la invención del automóvil, el cual llegó como solución al problema del transporte por tierra, de una forma más cómoda y práctica. Dicho problema no se resolvió hasta que el ingeniero alemán Karl Friedrich Benz creó el primer automóvil en 1885 \cite{cernuschi2005cuatro}, abriendo de esta forma, las puertas a un nuevo mundo para la investigación.\\

\par A partir de esta invención, lo que se ha buscado es mejorar su calidad, así como la experiencia en su utilización. En este ámbito, los vehículos autónomos buscan resolver los problemas más alarmantes de la conducción manual, por ejemplo: el número de accidentes generados por fallas humanas. Siendo en 1939  su primera aparición \cite{VeA}, el diseñador industrial Norman Bel Geddes, presentó en la feria de muestras Futurama, varios modelos de carreteras automáticas, que proporcionaban energía a vehículos eléctricos controlados por radio. Más tarde en la década de los 60, un equipo de investigadores alemanes construyó el primer carro completamente automatizado, el cual hizo uso de visión sacádica, cálculos probabilísticos y computación paralela, para resolver distintos problemas simultáneamente. Ya con las investigaciones un poco más avanzadas, varios países europeos decidieron enfocarse en la investigación de vehículos atonómos, consiguiendo de esta forma un gran progreso en temas de seguridad, confort e inteligencia. Estos avances se vieron reflejados en un vehículo construido por la compañia Mercedes Benz en 1994, el cual rocorrió más de 1.000 km en París logrando adelantar vehículos, y alcanzar velocidades cercanas a los 130 km/h \cite{carreno2012diseno}.\\  

\par De estos progresos se pueden destacar los sistemas avanzados de asistencia al conductor, \textit{ADAS (del inglés, Advanced Driver Assistance Systems}), que se enfocan en garantizar la seguridad y aumentar la comodidad al momento de conducir, con el fin de poder reducir el número de accidentes. Estos sistemas fueron introducidos por primera vez en 1986 \cite{ADAS}, en el que varios institutos de Europa, formaron la iniciativa Prometheus, buscando soluciones prácticas a problemas de tráfico urbano. Los sistemas ADAS emplean en su mayoría un conjunto de subsistemas enfocados a la interacción con el usuario, de los cuales se pueden nombrar: los sitemas de velocidad de crucero autónomo, sistemas de advertencia de conducción incorrecta, detector de cambio de carril, sistemas anticolisión, entre otros.\\ 

\par Por otro lado, se encuentran los Sistema de Transporte Inteligente, ITS (del inglés, \textit{Intelligent Transport System}), que se presentan como una combinación de distintos sistemas avanzados de información, comunicación y control aplicados a vehículos e infraestructuras \cite{rastelli2012agentes}. Al automatizar los vehículos en conjunto con las infraestructuras se obtiene como resultado una simplificación de problemas a resolver, ya que las mismas se pueden encargar de situaciones que esten fuera del alcance de los vehículos, como por ejemplo, la coordinacióncon otros vehículos en una intersección, así como también optimizar el cobro electrónico de peajes, el control de disponibilidad y ubicación de puestos en un estacionamiento, entre otros.\\  
  
\par Con el pasar del tiempo los vehículos autónomos han ido cobrando mayor importancia, cada vez más empresas, universidades e institutos se han dedicando al desarrollo y optimización de los mismos, un ejemplo son los institutos dedicados a las telecomunicaciones, como el instituto europeo de normas de telecomunicaciones, ETSI (del inglés,\textit{ European Telecommunications Standars Institute}), el cual diseñó los estándares europeos por los cuales se rigen las comunicaciones vehículares, lo que hace posible que se lleve a cabo de forma efectiva la conexión entre vehículos e infraestructuras.\\    

\par Otro ejemplo que se puede resaltar es la compañia, Toyota \footnote{http://www.toyota-global.com/innovation/smart\_mobility\_society/intelligent\_transport\_systems/} la cual desarrolló un  control de crucero adaptativo, ACC (del inglés, \textit{Adaptive Cruise Control}) mediante comunicaciones vehiculares, con el cual detectan de forma más precisa y ágil las aceleraciones y deceleraciones de los otros vehículos, logrando de esta forma mejorar en buena medida el ACC mediante radar. De la misma forma, la compañia Nissan \cite{VeA} mediante comunicaciones con infraestructura, ha logrado advertir a los vehículos sobre la proximidad a una intersección, así como también si los mismos se encuentran dentro del rango de velocidad permitido en la vía por la que circulan.\\

\par Sin embargo, la compañias que más destacan en esta área son Google \cite{tecnicocentro} y Tesla \cite{palin2012aerodynamic}, donde la primera ha automatizado distintos carros como el Toyota Prius, o el Audi TT, hasta el punto de poder recorrer las calles de California, sin necesidad de un conductor. Dentro de estas incorporaciones se encuentran: sensores LIDAR de rotación, que escanean a mas de 200 metros en todas las direcciones para generar un mapa preciso del entorno, sensores de estimación de posición, una cámara en el techo para detectar señales de tránsito, sensores de radar de Standar para determinar las posiciones de los objetos al rededor del vehículo, entre otros. De la misma forma, la segunda compañia con su modelo Tesla-S, logra combinar distintos radares, un sonar de 360 grados y cámaras, para conceguir una conducción autónoma, considerando varios aspectos, como lo son cambios de canales, estacionado autmoático, señales de tránsito, etc.\\

\par De esta forma los avances en los vehículos autónomos se pueden dividir en 6 niveles, definidos por la sociedad de ingenieros de automoción, SAE (del inglés, Society of Automotive Engineers) \cite{standard2014j3016}:
\begin{itemize}
\item Nivel 0, sin ninguna automatización: el conductor tiene el 100 \% del control del vehículo
\item Nivel 1, asistencias a la conducción: el vehículo ya asume ciertas tareas, como mantenimiento de carril,
\item Nivel 2, automatización parcial:  los carros ya peuden realizar acciones como, seguir a otros automóviles.
\item Nivel 3, automatización condicionada: los vehículos ya son capaces de pensar por sí mismo, logrando tomas deciciones apropiadas en distintos escenearios como cambio de carril, frenado de emergencia, etc.
\item Nivel 4, alta automatización: ya no es necesario la intervención del ser humano, el carro reaccionará de manera correcta siempre y cuando no se ponga en riesgo la salud de los ocupantes.
\item Nivel 5, automatización total: el vehículo es capaz de trasladarse a cualquier parte sin necesidad de volante, pedales, mandos, etc.\\ 
\end{itemize}

\par Actualmente, en su mayoría, los vehículos se encuentran entre los niveles 1 y 2, dentro de los cuales se encuentran los carros diseñados por la Toyota o Nissan, por poner un ejemplo. Siendo los de mayor nivel los de Google y Tesla, que se podría decir que ya están entre los niveles 3 y 4.\\

\par En pro de contribuir con la automatización de los vehículos, el presente proyecto busca desarrollar algoritmos inteligentes para realizar distintas maniobras cooperativas entre vehículos automatizados y semi-automatizados basadas en comunicaciones V2V, los cuales puedan ser implementados, tanto en entornos reales, como virtuales.

\section{Justificación y Planteamiento del Problema}

En la actualidad, más de 1,25 millones de presonas mueren cada año como consecuencia de accidentes de tránsito, y aproximadamente 50 millones sufren traumatismos no mortales, los cuales pueden llegar a producir alguna discapacidad \footnote{http://www.who.int/mediacentre/factsheets/fs358/es/}. Dichos siniestros son causados, en su mayoría, por la imprudencia del ser humano. Si no se toman medidas correctivas se espera que estas cifras tan alarmantes aumenten para el año 2030, de tal forma que se conviertan la séptima causa de muerte en el mundo. Es por eso que la organización de las naciones unidas, ONU (del inglés, \textit{Organization of United Nation}) adoptó \textit{La agenda 2030} para el desarrollo sostenible, donde se espera que para el 2020 se disminuyan estos números a la mitad, a través de dsitintos planes.\\ 

\par Para apoyar estas soluciones que se pretenden poner en práctica, el desarrollo e implementación de los ITS juegan un papel muy importante, ya que los mismos buscan solventar los fallos del ser humano, ya sea mediante acciones pasivas, como lo puede ser una simple notificación al conductor de alguna falla o infracción que este cometiendo, o mediante acciones activas, como lo puede ser tomar el control del vehículo, en caso de una emergencia. Para lograr que se puedan realizar efectivamente estas labores, cada uno de los sistemas integrados en los ITS deben de ofrecer el mejor rendimiento, es por esta razón que los estudios actualmente se centran en la mejora de dichos sistemas.\\  

\par De estos sistemas se pueden destacar los relacionados al área de las comunicaciones, los cuales son los encargados de conectar vehículos, peatones e infraestructuras entre si, y que de esta forma se puedan intercambiar distintos datos, como por ejemplo velocidad, posición, aceleraciones, etc, así como mensajes de adevertencia de siniestros, riesgos, entre otros. A su vez, los sistemas de comunicación dan paso a los sistemas cooperativos, los cuales poseen un extenso campo de aplicaciones, donde se pueden resaltar las que involucran maniobras con más de un vehículo.\\ 

\par El desarrollo de los ITS puede llegar a ser muy complicado y en casos muy peligroso, si se toma en cuenta el hecho de que actualmente las avances no solo estan enfocados para mejorar el desempeño del vehículo como individuo, si no como conjunto, es decir, estan más centrados en la mejora de la interacción de los mismos con su entorno. Es en este escenario donden entran centros de investigaicón como Tecnalia, que con su esfuerzos buscan formas más seguras de poder desarrollar estos sistemas, muestra de esto es el simulador Dynacar, el cual permite realizar la simulación de los vehículos en distintas situaciones, asegurándose de esta forma que el vehículo cumpla con las objetivos propuestos antes de ser probado en la realidad, reduciendo de esta forma los riesgos que pueda generar la prueba experimental.\\

\par Sin embargo, Dynacar es un simulador aún en desarrollo, lo que implica que carece de todos los componentes necesarios para probar todos los escenarios conocidos. Dentro de estas carencias se encuentra la falta de un sistema de comunicación que permita probar varias maniobras cooperativas en más de una computadora. Es por esta razón que el presente trabajo tiene como finalidad, desarrollar un sistema de comunicación de bajo coste que permita la realización de distintas maniobras que involucren más de un vehículo, las cuales puedan ser probadas primero en el simulador Dynacar, para luego ser llevadas a cabo en la realidad y más aún que permita la prueba de estas maniobras en un entorno combinado entre vehículos reales y vehículos virtuales.       

\section{Objetivos}

\subsection{Objetivo General}

Implementar un sistema de comunicación entre vehículo reales y virtuales, con el fin de que se puedan realizar distintas maniobras coopertativas en distintos ambientes de prueba. 

\subsection{Objetivos Específicos}
\begin{itemize}
	\item Revisar el estado del arte de los vehículos autónomos, haciendo énfasis en las comunicaciones entre infraestructuras y vehículos.
	\item Desarrollar e implementar un sistema de monitoreo de vehículos empleando el simulador Dynacar.
	\item Desarrollar e implementar un sistema de comunicación efectivo entre vehículos.
	\item Elaborar e implementar los respectivos controladores para las maniobras cooperativas.
	\item Probar las maniobras cooperativas en los distintos entornos.
	\item Realizar el informe final así como la debida documentación de cada bloque realizado.
\end{itemize}

\section{Estructura del Trabajo}

Habiendo realizado la respectiva introducción al problema a tratar en el proyecto, así como los objetivos de este, a continuación se presentarán 7 capítulos más con sus respectivos resúmenes.\\

\par En el capítulo 2, se presentará una descripición del instituto receptor, TECNALIA \textit{Research \& Innovation}, exponiendo su historia y una breve presentación del equipo de Automated Driving, con el cual se realizó este proyecto, dando a conocer sus objetivos y los vehículos que cuenta.\\

\par La revisión del estado del arte esexpuesta en el capítulo 3, donde se tocan distintos conceptos relacionados con temas de comunicación, y de vehículos autónomos, buscando una mejor comprensión del proyecto. Luego se habla sobre los sitemas de comunicación, tanto vehiculares, como comerciales. Seguidamente se presenta una descripción de los sistemas cooperativos, así como una lista de las maniobras cooperativas realizadas, dejando de  último descripción de la estrategia de control empleada, que en este caso fue la lógica borrosa.\\

\par En el capítulo 4, se hará una descripción del software utilizado, Matlab, así como también del simulador Dynacar, haciendo enfasis en su arquitectura y distintos componentes.\\

\par El capítulo 5, se describe el sistema de comunicación utilizado, en donde primero se analizarán las razones por las cuales se empleó el sistema de comunicación comercial, para luego describir los bloques realizados en Matlab para el simulador Dynacar, junto con su respectivo análisis de rendimiento.\\

\par En el capítulo 6, se presentarán las maniobras cooperativas realizadas, control de crucero adaptativo, ACC, el control lateral, Stop and Go, y el ACC junto con el control lateral. En el caso de las dos primeras se describirán los algoritmos de control, así como su implementación en Dynacar y con sus respectivas pruebas, mientras que, para las dos últimas solo se describirán las pruebas realizadas con dichas maniobras.\\

\par En el capítulo 7, se presentarán los resultados obtenidos al realizar las respectivas pruebas de cada maniobra en los entornos de prueba, los cuales son: Comunicación PC-PC, Vehículo-Vehículo y Vehículo-PC.\\

\par En el capítulo 8, se presentarán las conclusiones del trabajo, así como los trabajos a futuro que se puedan realizar teniendo como base el presentado.     

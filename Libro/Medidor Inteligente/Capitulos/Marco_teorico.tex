\chapter{Marco Teórico}\label{sec:Marco_Teorico}
\thispagestyle{empty}

\begingroup
\rightskip0.5cm
\small

\endgroup

\section{Teorema de muestreo}

\section{Acondicionamiento de la señal}
\par Existen maneras de cuantificar un fenómeno físico con parámetros eléctricos
mediante la utilización de sensores electrónicos, sin embargo, la salida de estos
no siempre es apta para realizar un procesamiento de la data, por lo que es necesario
acondicionar la señal adquirida mayormente mediante distintos circuitos eléctronicos
que ofrecen funciones como: ajuste de ganancia y nivel, filtrado, adaptación de impedancia
y aislamiento.

\subsection{Ajuste de ganancia y nivel}
\par En busca de lograr una amplificación o una atenuación es necesario conocer
los amplificadores operacionales (OPAM), estos dispositivos electrónicos son capaces
de alterar la señal electrica en su entrada, multiplicando su valor por una ganancia
determinda por la configuración externa. Por ejemplo, la configuracion no-inversora:

(IMAGEN DE LA CONFIGURACION NO INVERSORA)

\par Logrando transformar una señal que oscila entre -10 y +10 V a una entre -2.5 y 2.5 V, estableciendo una ganancia de 0.25.

\par Por otro lado comunmente los dispositivos electrónicos no son capaces de procesar
señales bipolares, por lo tanto se debe hacer un ajuste del nivel a las señales bipolares.
Este ajuste se puede lograr mediante la utilización de un divisor de tensión resistivo
mostrada a continuación:

(IMAGEN DE LA CONFIGURACION DEL AJUSTE DE NIVEL)

\par Con esta configuración es posible llevar una señal electrica oscilante entre -2.5 y 2.5 V
 a una entre 0 y 5 V.

\subsection{Filtrado}
\par
\subsection{Adaptación de impedancia}
\subsection{Aislamiento}
\par Por motivos de seguridad es necesario separar 







\subsection{Convertidor analógico-digital}

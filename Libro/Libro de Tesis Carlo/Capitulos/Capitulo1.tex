%Descripción de la empresa
\chapter{Descripción de la Empresa}
\thispagestyle{empty}

El INRIA (Instituto Nacional de Investigación en Informática y Automática, del francés: Institut National de Recherche en Informatique et en Automatique), es un instituto de investigación francés fundado en 1967. Éste posee 8 sedes alrededor de Francia, es el único centro público francés completamente dedicado a la Informática y la Automática. \\

El INRIA es hoy en día uno de los centros de investigación de tecnología de punta, liderando y formando parte de un sin número de proyectos en el ámbito europeo, así como en el internacional. Dentro de él se encuentran un gran número de grupos de investigación, siendo uno de ellos el equipo IMARA (Informática, Matemáticas y Automática para la Ruta Automatizada, del francés: Informatique, Mathématiques et Automatique pour la Route Automatisée), el cual será el equipo receptor del presente trabajo.

\section{Reseña histórica}

\noindent
\textbf{EL INRIA}

Con más de cuarenta años desde su fundación, ha abordado las ciencias de la computación desde su infancia, hasta el dominio digital del presente. Su historia puede ser resumida en cuatro períodos importantes a saber a continuación:\\

%\begin{itemize}
\noindent
\textbf{1967-1973: En búsqueda del rumbo.}

La creación del INRIA (1967), fue un símbolo de la política proactiva francesa presente en el gobierno del presidente De Gaulle. En estos tiempos, el estado francés buscaba desarrollar tecnologías de punta y su posterior producción. Lo anterior promueve al estado francés a desarrollar un instituto, con capacidad competitiva en el ámbito informático y de control, el cual tendría además, la capacidad de educar al país en informática y automática.\\

En sus inicios, su nombre era IRIA (Instituto de investigación en Informática y automática, del francés: Institut de Recherche en Informatique et en Automatique), siendo incluso hoy en día una institución que sirve de puente entre el sector público y la industria. Pensado para ser la mano derecha del CII (Compañía Internacional para la Informática, del francés: Compagnie Internationale pour l’informatique), el IRIA se crea bajo el mando de Michel Laudet.\\

Su carácter internacional salió a relucir de inmediato, gracias a la organización de diferentes conferencias internacionales, a las cuales eran invitados los grandes pensadores del mundo de la informática y las matemáticas aplicadas. Siendo una de sus prioridades la educación dentro de los campos previamente mencionados, se establecen varias escuelas de verano y centros de cursos.\\

\noindent
\textbf{1974-1979: El largo camino a la madurez.}

El período de 1974-1979 vio tanto la maduración del centro, como su modificación en algunos proyectos internos. Para este punto es importante mencionar a Jacques-Louis Lions, quien para 1973 era jefe de Laboria (Laboratorio de Investigación en Informática y Control, del inglés: The computer science and control research laboratory).\\

Gracias a Laboria, laboratorio que formaba parte del IRIA, se realizaron presentaciones en Francia, Europa y Estados Unidos, lo cual colocó al IRIA a la vanguardia de las investigaciones en el campo computacional. \\

En 1979, la descentralización francesa amenaza con desaparecer el IRIA, esto debido a que se pensaba absorber al IRIA con algún instituto como el IRISA (Instituto de Ciencias de la Computación y Sistemas Aleatorios, del francés: Institut de recherche en informatique et systèmes aléatoires). Sin embargo, al asumir Jacques-Louis Lions el cargo de CEO, el instituto fue conservado en Rocquencourt (París) y, el 27 de diciembre de 1979, al nombre le fue agregada una “N”, conocido de ese momento en adelante como el INRIA.\\

\noindent
\textbf{1980-2000: La rápida expansión del INRIA.}

El cambio más grande en la historia del INRIA se hace realidad al llegar Jacques-Louis Lions a la presidencia en 1980. El instituto aún carecía de los recursos necesarios, sin embargo, un modelo de desarrollo de tecnologías de alta calidad, para el desarrollo de la industria, fue aplicado exitosamente.\\

 Gracias a la creación de una red europea de investigadores, el INRIA jugó un papel importante en el desarrollo del internet en Europa. Ya para este momento, luego de innumerables ensayos y errores, el INRIA poseía un claro rumbo, y una muy alta reputación internacional.\\

En este período, fueron construidos varios centros alrededor de Francia, gracias a la descentralización. Entre ellos tenemos:

\begin{itemize}
\renewcommand\labelitemi{$\circ$}

\item Irisa y la unidad de investigación en Rennes en 1975.

\item La unidad de investigación Sophia-Atipolis en 1983.

\item La unidad de investigación en Lorraine / Loria en 1986.

\item La unidad de investigación Rhone-Alpes en 1992.

\item La unidad de investigación Futurs desde el 2003, con miras a tres unidades nuevas en Bordeaux, Lille y Saclay.

\end{itemize}

Dentro del marco del instituto, la tecnología innovadora era y es una constante, incurriendo constantemente en actividades que incluyen la creación de patentes, asociaciones con entes del sector industrial, el funcionamiento de consorcios y el soporte para la innovación en los negocios. Sin duda, el número de colaboración internacional creció rápidamente.\\

\noindent
\textbf{INRIA llega al siglo 21}

Para el 2007, INRIA llega a su 40vo aniversario, con una expansión notable. Entre 1999 y 2009, su fuerza de trabajo se vio duplicada. Tres nuevos centros de investigación fueron añadidos a los 5 ya existentes, estos fueron los centros de Saclay, Bordeaux y Lille. Fuertemente arraigado en el ámbito de la industria local y la educación, el INRIA busca actualmente un mayor desenvolvimiento en cuanto al ámbito de investigación europeo. \\

Teniendo en cuenta que el futuro de nuestras sociedades yace en el mundo digital, el INRIA ataca problemas cruciales en el desenvolvimiento de las sociedades de hoy en día. Siendo una de las 10 primeras instituciones contribuyentes al Marco de Programas Europeos para la Investigación y el Desarrollo (FPRD, del inglés: European Framework Programmes for Research and Developement). \\

Finalmente es importante resaltar la mirada al futuro del INRIA con su plan estratégico 2008-2012, el cual se basa en superar los retos del siglo 21. INRIA está en constante evolución y fortalecimiento gracias a asociaciones con diferentes disciplinas científicas y el mundo económico, viéndose especialmente en Francia, Europa, Estados Unidos y países emergentes como China, India, Suramérica y África.\\

\noindent
\textbf{Equipo IMARA}

El equipo IMARA, creado en el 2008, está orientado a la investigación en ITS, (Sistemas de Transporte Inteligentes, del inglés: Intelligent Transportation Systems). Su estructura organizacional se ve conformada por un equipo “Horizontal”, liderado actualmente por el PhD. Prof. Fawzi Nashashibi, siendo su objetivo inicial la agrupación de conocimientos del INRIA y el mundo, donde dichos conocimientos son aplicables a la Automática en vehículos autónomos y al concepto LaRA \footnote{imara.inria.fr/lara} (La ruta automática, del francés: La Route Automatisée).

\section{Misión}

La principal misión del INRIA se enfoca en la excelencia científica y la transferencia de tecnología\footnote{http://www.facebook.com/Inria.fr/info} (del francés: Les missions d'Inria concernent l'excellence scientifique et le transfert technologique). Sin embargo, las palabras de Michael Cosnard, presidente y CEO (Jefe Ejecutivo, del inglés: Chief Executive Officer) del INRIA pueden dar un mejor entendimiento a la frase anteriormente expuesta:\\

$"$Conceiving the infrastructures of the future sustainable digital society is one of our priorities. Energy consumption related to these technologies, their environmental impact and the potential for ensuring a sustainable development of our societies are key issues that mobilize our research teams. With our experience of partnership with numerous public and private leaders of research and innovation, we are convinced that the joint efforts within the GreenTouch Initiative will provide genuine solutions to these challenges.$"$\\

Michel Cosnard, Chairman and CEO, INRIA.\footnote{http://www.greentouch.org}\\

En español:\\

$"$Concebir las infraestructuras del futuro de la sociedad digital sostenible es una de nuestras prioridades. El consumo de energía relacionado con estas tecnologías, su impacto ambiental y la posibilidad de garantizar un desarrollo sostenible de nuestras sociedades son cuestiones clave que movilizan nuestros equipos de investigación. Con nuestra experiencia de colaboración con numerosos líderes públicos y privados de investigación e innovación, estamos convencidos de que los esfuerzos conjuntos de la Iniciativa GreenTouch proporcionará soluciones reales a estos desafíos.$"$\\

Michael Cosnard, Presidente y CEO, INRIA.

\section{Objetivo del grupo receptor: IMARA}

Mejorar el transporte terrestre en términos de seguridad, eficiencia y confort; basado en investigaciones orientadas a tecnologías de ayuda al conductor, hasta alcanzar el completo manejo automatizado\footnote{https://team.inria.fr/imara/}. Esto se logra gracias a la coordinación y transferencia de todas las investigaciones hechas en el INRIA, las cuales pueden ser aplicadas al concepto LaRA.

\section{Estructura organizacional y equipo IMARA}

\section{Documentos y Publicaciones}


% ***************************************************
% CARÁTULA DEL TRABAJO DE PASANTÍA
% ***************************************************
% \begin{titlepage}
\thispagestyle{empty}
    \begin{figure}[!ht]
        \begin{center}
            \includegraphics[scale=0.3]{../Imagenes/cebolla.jpg}
        \end{center}
    \end{figure}

    \vspace{-1.cm}

    \begin{center}
		\begin{large}
			UNIVERSIDAD SIMÓN BOLÍVAR\\
		\end{large}
    		\textbf{DECANATO DE ESTUDIOS PROFESIONALES}\\
	  		\textbf{COORDINACIÓN DE INGENIERÍA ELECTRÓNICA}\\

    \end{center}


    %\begin{center}
        %TRABAJO DE GRADO\\
    %\end{center}

    \vspace{4cm}
    \begin{center}
    %\large {
    \textbf{OPTIMIZACIÓN DE UN MEDIDOR ELÉCTRICO INTELIGENTE BIFÁSICO}\\
    %}
    \end{center}

    \vspace{3cm}
    \begin{center}
        Por:\\
        Andrés Suárez Figueroa
    \end{center}

		    \vspace{1cm}
    \begin{center}
        \textbf{PROYECTO DE GRADO}\\
				Presentado ante la Ilustre Universidad Simón Bolívar\\
				como requisito parcial para optar al Título de\\
				Ingeniero Electrónico\\
    \end{center}

    %\vspace{\stretch{1}}
	\vspace{1cm}
    \begin{center}
		\textbf{Sartenejas, en algún momento de 2018}
    \end{center}
    \pagebreak
% \end{titlepage}
